%
% File eamt11.tex, adapted from eamt10.tex
%
% Contact: vincent@ccl.kuleuven.be

%%% To ease future customizations, various replaceables have been paramaterized
%%% as listed in the newcommands section

\documentclass[11pt]{article}
\usepackage[utf8x]{inputenc}
\usepackage{eamt11}
\usepackage{url}
\usepackage{times}
\usepackage{multirow}
\usepackage[small,bf]{caption}
\usepackage{latexsym}
\usepackage{gb4e}
\usepackage{linguex} %think it has to be the last package

\setlength\titlebox{6.5cm}    % Expanding the titlebox

\newcommand{\confname}{EAMT 2011}
\newcommand{\website}{http://www.eamt2011.eu/}
\newcommand{\contactname}{the programme chairs Mikel Forcada \& Heidi De Praetere}
\newcommand{\contactemail}{mlf@dlsi.ua.es} 
\newcommand{\conffilename}{eamt11}
\newcommand{\downloadsite}{http://www.eamt2011.eu/}
\newcommand{\paperlength}{$8$ (eight)}
\newcommand{\shortpaperlength}{$4$ (four)}

\title{Rapid rule-based machine translation between Dutch and Afrikaans}

\author{Pim Otte\\
  Mendelcollege\\
  Pim Mulierlaan 4\\
  2023 WC Haarlem\\
  {\tt 5666@mendelcollege.nl}  \And
  Francis M. Tyers\\
  Dept. Lleng. i Sist. Inform.\\
  Universitat d'Alacant\\
  E-03070 Alacant \\
  {\tt ftyers@dlsi.ua.es}}

\date{}

\begin{document}
\maketitle
\begin{abstract}
 This paper describes the design, development and evaluation of a machine
 translation system between Dutch and Afrikaans developed over a period of
 around a month.
\end{abstract}

\section{Introduction}

The Dutch language is West-Germanic and is spoken by nearly 23 million people, which are mostly from the Netherlands and Flanders, the Dutch-speaking part of Belgium and a minority lives in former colonies of the Netherlands, such as Surinam, Aruba and the Netherlands Antilles. The Dutch language as it is today, started developing in the 16th century in the major trade cities, such as Amsterdam and Antwerp. \cite{Shetter:02}  Afrikaans is spoken by at least 5 million people, mainly in South-Africa. Afrikaans is a variety of Dutch that probably originates from the Dutch that was spoken by the Dutch colonists of the Cape Colony, from the pidginisation in the mouths of  the non-white inhabitants of the Cape or from a combination of these two factors. In 1925 Afrikaans replaced Dutch as an official language, to be the joint official language together with English. Currently Afrikaans is one of the eleven national languages. \cite{Donaldson:93}
In this paper we will describe the development of apertium-af-nl, an Afrikaans and Dutch machine-translation system based on the Apertium platform. Since Afrikaans and Dutch are largely mutually intelligible, this machine translation system focuses on dissemination, the production of text for the purpose of being post-edited and then being published. Previous work in Afrikaans and Dutch machine translation is described in \cite{Huyssteen:09} 
The paper is laid out as follows, firstly, we will describe the reuse and creations of resources. We will then discuss several grammatical features of Afrikaans and Dutch and how these were incorporated in the machine-translation system. We will then present a section in which the system is evaluated. Finally, we will discuss the system and future work that could be done.


\begin{itemize}
\item languages, machine translation, existing work
\item rationale: no free corpora of dutch--afrikaans.
\end{itemize}

\section{Method}

The system is based on the Apertium machine translation platform.\footnote{\url{http://www.apertium.org}} The
platform was originally aimed at the Romance languages of the Iberian
peninsula, but has also been adapted for other, more distantly related,
language pairs. The whole platform, both programs and data, are licensed
under the Free Software Foundation's General Public
Licence\footnote{\url{http://www.fsf.org/licensing/licenses/gpl.html}} (GPL)
and all the software and data for the 25 supported language pairs (and the
other pairs being worked on) is available for download from the project
website.

\begin{itemize}
\item development time, how apertium works (fig. 1)
\end{itemize}

\subsection{Existing resources}

One existing resource was reused with very little modification, that is the 
morphological transducer for Afrikaans created during a separate project on English--Afrikaans
machine translation. Some changes were made. The structure of verb entries
was wholly revised, and infrequent words were removed.

\subsection{Resources created}

\subsubsection{Dutch morphological transducer}

There are a number of existing morphological analysers for Dutch, based on ...

We chose to create a new analyser because... (tagset, gender, analysis/generation, free software)

The open categories (nouns, verbs, adjectives, adverbs) for the Dutch 
morphological analyser were extracted semi-automatically from 
Wiktionary,\footnote{\url{http://www.wiktionary.org}} a free online 
dictionary that often includes inflectional information. In the case 
of Dutch nouns it often (although not always) gives the gender and the 
plural and diminutive forms (see for example Figure~\ref{fig:wikt1}).
The category {\small {\tt Dutch nouns}} has a total of 10,610 entries,
while the corresponding category on the Dutch Wiktionary has 13,176 entries.

Closed categories were added by hand based on a grammar of Dutch \cite{Shetter:02}.

\begin{figure*}
\centering
\begin{small}
\begin{tabular}{|l|}
\hline
{\large Dutch} \\
~\\
{\bf Etymology}\\
~\\
~{\em hoofd-} (``main, head'') + {\em stad} (``city'')\\
~\\
{\bf Pronunciation}\\
~\\
{\bf Noun}\\
~\\
{\bf hoofdstad} m. ({\em plural} hoofdsteden, {\em diminutive} hoofdstadje, {\em diminutive} plural hoofdstadjes) \\
~\\
~~~~1. capital city \\

\hline
\end{tabular}
\end{small}
\caption{English language Wiktionary article for Dutch \emph{hoofdstad} `capital city' 
    {\small \url{http://en.wiktionary.org/wiki/hoofdstad}}}
\label{fig:wikt1}
\end{figure*}

\subsubsection{Bilingual dictionary}

\begin{itemize}
\item wikipedia, wiktionary, cognates
\end{itemize}

\subsubsection{Transfer rules}

%%Dutch cannot use the present tense in past situations (not even in narrative style)

\begin{itemize}
\item Concordance of article and complement
\item Concordance between subject and verb
\item Removal of negation scope marker
\end{itemize}

\begin{itemize}
\item Addition of negation scope marker
\end{itemize}


\subsubsection{Compound words}

%% give some statistics for what percentage of words are compounds in dutch/afrikaans
%% give some justification for only doing noun-noun compounds (e.g. more frequent, easier to identify??)

Both Afrikaans and Dutch are languages in which nouns combine very
productively into compounds. For example the words {\em infrastruktuurontwikkelingsplan}
`infrastructure development plan' and {\em personeelverminderingsprosedure}
`personnel protection procedure'. As it is impractical to introduce
all compound words into the lexicons, compound word analysis is performed on
all unknown words. The analysis process works longest-match left-to-right
using the same transducer as is used for morphological analysis.
Results are restricted by two special
symbols which do not appear in the output {\small {\tt compound-L}} and {\small {\tt compound-R}}.
The {\small {\tt compound-L}} symbol is used for forms that can only appear on the
left side (e.g. surface form) of a compound, where {\small {\tt compound-R}} is
used for forms that can either appear in a compound, or end it.
Epenthetics, that is linking letters that occur between compound words
are also taken care of heuristically in this way. For example the -s-
in ontwikkelingsplan or the -e- in EXAMPLE.


\begin{figure*}
\begin{small}
\begin{verbatim}
die personeelverminderingsprosedure

^die<det><def><sg>$
^personeel<n><sg><cmp>+vermindering<n><sg><cmp>+prosedure<n><sg>$

^de<det><ind><mf><sg>$
^personeel<n><nt><sg><cmp>$^terugdringing<n><f><sg><cmp>$^procedure<n><mf><sg>$

de personeelterugdringingsprocedure
\end{verbatim}
\end{small}
\end{figure*}

There are some limitations to this method. For example although
both {\em ontwikkeling-} and {\em ontwikkelings-} can be analysed as an internal part
of a compound, only one of them can be generated. Which one will be generated
is decided based on the inflectional paradigm to which the word belongs. For
example {\small {\tt ontwikkeling$<$n$><$sg$><$cmp$>$}} will produce {\em ontwikkelings-} in generation,
where {\small {\tt personeel$<$n$><$sg$><$cmp$>$}} will produce personeel.

\subsubsection{Separable verbs}

Another feature of Afrikaans and Dutch is separable verbs, for example
the Dutch {\em opslaan} `to save'. This can appear in the following
forms {\em opslaan}, {\em sla op}, {\em opgeslagen}. Additionally the two constituent
parts of the verb in {\em sla op}, the verb itself {\em sla} and the particle
{\em op} may be separated by a word or phrase, {\em Ik sla mijn bestanden op}
 `I save my files'.

The following cases are supported,

\begin{itemize}
\item Infinitive: opslaan $\rightarrow$ stoor 
\item Participle: opgeslagen $\rightarrow$ gestoor
\item Non-separated: Ik ga uit naar het strand vandaag $\rightarrow$
\item Subordinate: Toen ik snel mijn bestanden opsloeg $\rightarrow$ 
\end{itemize}

Verbs separated by a word or phrase are currently translated word-for-word,
so the particle and verb are translated. This causes a problem when the
verb is not constructed equally in Afrikaans and Dutch. Also, when one part
of the verb, for example in {\em aankondig} `to announce' does not exist as
a stand-alone verb, {\em kondig} is not a word. Thus {\em ... kondig ... aan} `'
cannot be analysed currently.

A module is under development to handle separable verbs, but is currently
in the prototype stage.

There are currently 484 separable verbs defined in the bilingual
dictionary. Of these, 439 are separable in both languages, 33 are
separable in Afrikaans but not in Dutch, and 12 are separable in
Dutch but not Afrikaans.

\subsection{Current status}

Dutch monolingual: 6,031
Afrikaans monolingual: 6,401
Bilingual: 5,965

\section{Evaluation}

The system was evaluated in five ways. The first was the 
coverage\footnote{Here coverage is defined as \emph{na\"ive coverage}, 
that is for any given surface form at least one analysis is returned. This 
may not be complete.} of the system. The second was an evaluation of the 
compound analysis part of the system -- new with respect to other 
Apertium language pairs. The third was the word error 
rate (WER) of the translations produced when comparing with a 
corrected sentence. The fouth was an analysis of the errors found by the second
evaluation and finally a comparative evaluation with existing systems.

\subsection{Coverage}

Lexical coverage of the system is calculated over the Afrikaans and Dutch Wikipedias:
Both corpora were split into four sections and coverage calculated over each of the 
sections in order to calculate the standard deviation.

\begin{table}
  \begin{center}
  \begin{tabular}{|l|r|r|}
   \hline
   {\bf Corpus}           & {\bf Tokens}    & {\bf Coverage}\\
   \hline
   {\tt af} Wikipedia     & 2,926,943       & 82.1\% $\pm$ 0.8 \\
   \hline
   {\tt nl} Wikipedia     & 18,569,183      & 80.5\% $\pm$ 0.7 \\
   \hline
  \end{tabular}
    \caption{Na\"ive vocabulary coverage for the two morphological analysers.}
    \label{table:coverage}
  \end{center}
\end{table}

\subsection{Compound words}

In order to test the accuracy of the word compounding/decompounding strategy
we tested two lists of words which received compound analyses from 
the Wikipedia. This test was only conducted in the Afrikaans$\rightarrow$Dutch
direction, but we expect similar results in the other direction. The first
set of sentences was constructed by taking the highest frequency 1,000 words which received 
a compound analysis, the second was by taking a list of all the words and selecting
1,000 pseudo-randomly.\footnote{Using the Unix {\small {\tt unsort}} program.} A total 
6,866 unknown words from the corpus received a compound analysis.

\begin{table}
  \begin{center}
  \begin{tabular}{|l|r|r|}
   \hline
   {\bf Corpus}    & {\bf Corr. Seg.}    & {\bf Corr. Trans.}\\
   \hline
   top-1,000       & 914                 &  776 \\ 
   \hline
   random-1,000    & 957                 &  801 \\ 
   \hline
  \end{tabular}
    \caption{Compound word accuracy in analysis and translation.}
    \label{table:compounds}
  \end{center}
\end{table}

We include results for both correct segmentation (meaning the word was decompounded 
correctly) and correct translation (meaning the word was translated correctly). This allows
us to take into account the {\em free ride} phenomenon, whereby an incorrect analysis
may lead to a correct translation. There were 19 free rides in the top-1,000, and 5 free 
rides in the random-1,000.

\subsection{Quantitative}

The translation quality was measured using two metrics, the first was word error rate (WER), and the 
second was position-independent word error rate (PER). Both metrics are based on the Levenshtein 
distance \cite{Levenshtein:65} and were calculated for each of the sentences using the 
{\small \texttt{apertium-eval-translator}} tool.\footnote{\url{http://sourceforge.net/project/showfiles.php?group_id=143781&package_id=206517}; Version 1.0, 4th October 2006.} Metrics based on word error rate were chosen as to be able to compare 
the system against systems based on similar technology, and to assess the usefulness of the 
system in a real setting, that is of translating for dissemination. 

Two sets of 100 sentences were selected pseudo-randomly from Wikipedia. The first set contained 
no unknown words, whereas the second set could contain unknown words. This is to give an idea
of the performance of the system in `ideal' and `realistic' settings.

\begin{table}
  \begin{center}
  \begin{tabular}{|l|c|r|r|}
   \hline
   {\bf Corpus}                & {\bf Dir.}  & {\bf WER}    & {\bf PER}\\
   \hline
   \multirow{2}{*}{ideal}      &  {\small {\tt af-nl}}      & 14.65 $\pm$ 1.5        &  14.17 $\pm$ 1.31 \\ 
                               &  {\small {\tt nl-af}}      & 23.41 $\pm$ 1.24     & 22.96 $\pm$ 1.26 \\
   \hline
   \multirow{2}{*}{realistic}  &  {\small {\tt af-nl}}      &              &  ~ \\ 
                               &  {\small {\tt nl-af}}      &              & ~ \\
   \hline
  \end{tabular}
    \caption{Accuracy for the two test corpora as measured by Word Error Rate 
        and Position-independent Word Error Rate.}
    \label{table:quan}
  \end{center}
\end{table}


\subsection{Qualitative}

In order to inform ourselves of where the effort could be expanded in order to improve the 
system we undertook a qualitative evaluation by reviewing the translation errors from the Afrikaans
to Dutch direction and categorising them as in Table~\ref{table:qual}. An example of each 
of the kind of error is found below.

\begin{table}
  \begin{center}
  \begin{tabular}{|l|c|r|r|}
     \hline
     {\bf Error type}    & {\bf Count} & {\bf Percentage of total} \\
     \hline
     Unknown word        & 147         & \\
     Morphology          & 3           & \\
     Disambiguation      & 106         & \\
     Multiword           & 6           & \\
     Syntactic transfer  & 263         & \\
     Polysemy            & 23          & \\
     Compounding         & 6           & \\
     Separable verb      & 3           & \\
     \hline
     Total               &             & \\
     \hline
  \end{tabular}
    \caption{Contribution to total error by type}
    \label{table:qual}
  \end{center}
\end{table}


\subsubsection{Unknown word}

\ex. \label{ex:exist} 
    Hierdie besetting is in 1721 met die Verdrag van Nystad erken. \\
    Deze bezetting is in 1721 met het Verdrag van *Nystad *erken. \\
    Deze bezetting is in 1721 met het Verdrag van Nystad erkend. \\

\subsubsection{Morphology}

\ex. \label{ex:exposs} 
    Pi is ook bekend as Archimedes se konstante (nie dieselfde as Archimedes se getal nie) en Ludolph se getal. \\
   *Pi is ook bekend als *Archimedess *konstante (niet #dezelfde als *Archimedess *getal ) en *Ludolphs *getal. \\
    Pi is ook bekend als Archimedes' constante (niet dezelfde als Archimedes' getal ) en Ludolphs getal.\\

This is an issue with possesive "se" from Afrikaans. In Dutch this can be translated with a possesive "s" added to the preceding noun,
 or a construction using the preposition "van". However, names or words that end in -s, -x, or generally, end in a sound like -s,
 cannot take another -s and take an apostrophe instead. A similar situation could erise if the preceding word ends in a vowel,
 in which case the word takes -'s instead of -s.

\subsubsection{Disambiguation}

\ex. \label{ex:exdisam} 
    In 1843 verklaar die Britte die stad tot 'n kolonie. \\
    In 1843 verklaren de Britten de stad tot een kolonie. \\
    In 1843 verklaarden de Britten de stad tot een kolonie.\\

In this example, in the Afrikaans sentence, the verb "verklaar" could be tagged as being one of present tense, past tense,
 infinitive or past participle. It has been tagged as present tense, which led to a translation which is not correct in Dutch, 
because the Dutch language does not allow the use of the present tense if an event is in the past,
 even if the sentence is in narrative style, like in this example.

\subsubsection{Multiword}

\ex. \label{ex:exmw} 
    In die laaste dekades het ook Duitse argitekte begin om projekte dwarsoor die wêreld aan te pak. \\
    In de laatst decenniën heeft ook Duitse architecten beginnen om projecten overdwars de wereld aan te pakken. \\
    In de laatste decenniën zijn ook Duitse architecten begonnen om projecten over de hele wereld aan te pakken. \\

This example is causing problems because it is hard, if not impossible to catch the meaning of the Afrikaans "dwarsoor" in one Dutch word.
 An appropriate multiword could fix this, but might cause additional issues with the article of "wereld" as that is included in the phrase "over de hele"

\subsubsection{Syntactic transfer}

\ex. \label{ex:exsgpl} 
    Hier volg 'n lys van hoofstede. \\
    Hier volgen een lijst van hoofdsteden. \\
    Hier volgt een lijst van hoofdsteden. \\

Here the plural verb does not match the singular subject, the noun "lijst".
 This could be solved by identifying the subject of the sentence and matching the plurality of the verb with it.

\ex. \label{ex:exhavebe} 
    Die belangrikste rol wat die vroue egter in die stryd teen apartheid gespeel het, was die beweging teen die uitreiking van pasboeke vir vroue. \\
    De belangrijkste rol wat de vrouwen echter in de strijd tegen apartheid gespeeld heeft, was de beweging tegen de uitreiking van pasboeken voor vrouwen. \\
    De belangrijkste rol die de vrouwen echter in de strijd tegen apartheid gespeeld hebben, was de beweging tegen de uitreiking van pasboeken voor vrouwen. \\

Afrikaans uses the verb "hê" (to have) with all past participles, whereas Dutch uses the verb "zijn" (to be) in cases of, amongst others, verbs that imply movement. This could be fixed by tracking past participles and their auxiliary verbs.

\subsubsection{Polysemy}

\ex. \label{ex:expolysem} 
    Analitiese chemie is die analise van materiaalmonsters om sodoende hul chemiese samestelling en struktuur te verstaan.
    Analytische chemie is de analyse van materiaalsteekproeven om zodoende hun chemische samenstelling en structuur te verstaan.
    Analytische chemie is de analyse van materiaalmonsters om zodoende hun chemische samenstelling en structuur te bepalen.

This sentence has two errors due to polysemy. The Afrikaans "monster", here as plural as the right part of a compound, 
can mean either steekproef or monster (both mean sample in English). However, the former is generally used in the context of statistical research, 
while the latter is used for samples of substances, or samples from animals. The other error due to polysemy is "verstaan".
 This Afrikaans word can be translated as "verstaan" in Dutch, but this means "to catch" (as in "to hear", "to understand"), 
which is clearly wrong in this example. The correct translation here is "bepalen" (to determine) 

\subsubsection{Compounding}

\ex. \label{ex:excompsplit} 
    Die graanopbrengs per hektaar is laer as in Middel-Europa. \\
    De *graanopbrengs per *hektaar is lager als in Middel-Europa. \\
    De graanopbrengst per hectare is lager dan in Midden-Europa. \\

The error due to compounding in this example is "Middel-Europa". This word has been translated by splitting it up and translating the seperate parts.
 However, while normally the Afrikaans "Middel" can be translated correctly as "middel", in the case of geographical names, the only correct translation is "Midden".

\ex. \label{ex:excomphyphen} 
    Die motornywerheid is die ekonomiese basis van Oshawa, en die stad huisves die grootste Kanadese monteerplase en hoofkwartier van General Motors, wat oorspronklik in die jaar 1876 as McLaughlin Carriage Company gestig is. \\
    De autoindustrie is de economische basis van *Oshawa, en de stad *huisves de grootste Canadezen *monteerplase en hoofdkantoor van *General Auto's, wat oorspronkelijk in het jaar 1876 als *McLaughlin *Carriage *Company gesticht is. \\
    De auto-industrie is de economische basis van Oshawa, en de stad huisvest de grootste Canadeze monteerplaatsen en het hoofdkantoor van General Motors, wat oorspronkelijk in het jaar 1876 als McLaughlin Carriage Company gesticht is. \\

The error in this example is due to a specific rule in the Dutch language in regards to compounds. If a compound is built-up from two words as such that 
the two vowels around the splitting point constitute a sound on their own, which means the word could be mispronounced,
 a hyphen should be used to distinquish the different parts of the compound. This phenomenom is known in Dutch as "klinkerbotsing". 
\subsubsection{Separable verb}

\ex. \label{ex:exsepverb} 
    Twee jaar later ruk die situasie in die land onder die indruk van massabetogings hand uit. \\
    Twee jaar later *ruk de situatie in het land onder de indruk van *massabetogings hand uit. \\
    Twee jaar later loopt de situatie in het land onder de invloed van massabetogingen uit de hand. \\

This is a classic example of a seperable verb which is not recognised as such. The Afrikaans "ruk ... hand uit" corresponds with
 the Dutch seperable verb (and expression) "loopt ... uit de hand" However, "ruk" (to pull) in itself could never be translated as "lopen" (to walk). 
Solving this is a significant MT challenge and is not easily fixable.  

\subsection{Comparative}

% compare against google
% part of the problem with this: google may use some of the sentences that appear in our corpus for training
% not possible 

% tried to compare against d2ac, but were not able to get access to their evaluation texts.

\section{Discussion}

\subsection{Future work}

\section*{Acknowledgements}

Development of this system was partially supported by the Google Code-in,
a contest to introduce pre-university students to contributing to open-source
software.

% \bibliography{\confname}

\begin{thebibliography}{}

\bibitem[\protect\citename{Shetter and Ham}2002]{Shetter:02}
Shetter, William~Z. and Ham, Esther.
\newblock 2002.
\newblock {\em Dutch: An Essential Grammar}, 9th~edition.
\newblock Routledge, Oxford.

\bibitem[\protect\citename{Donaldson}1993]{Donaldson:93}
Donaldson, Bruce C.
\newblock 1993.
\newblock {\em A grammar of Afrikaans}
\newblock Walter de Gruyter, Berlin

\bibitem[\protect\citename{van Huyssteen and Pilon}2009]{Huyssteen:09}
van Huyssteen, Gerhard and Pilon, Suléne.
\newblock 2009. 
\newblock Rule-based Conversion of Closely-related Languages: A Dutch-to-Afrikaans Convertor. 
\newblock {\em Proceedings of the 2009 Conference of the Pattern Recognition Association of South Africa}, Stellenbosch, SA
\newblock 23--28.

\bibitem[\protect\citename{Levenshtein}1965]{Levenshtein:65}
Levenshtein, Vladimir.
\newblock 1965. 
\newblock Binary codes capable of correcting deletions, insertions, and reversals
\newblock {\em Doklady Akademii Nauk SSSR}
\newblock 845--848.


\end{thebibliography}

\end{document}
